\appendix
\renewcommand{\thesection}{\Alph{section}}

\addtocontents{toc}{
  \protect\renewcommand{\protect\cftsecpagefont}{}%
  \protect\renewcommand{\protect\cftsubsecpagefont}{}%
  \protect\renewcommand{\protect\cftsecafterpnum}{}%
  \protect\renewcommand{\protect\cftsubsecafterpnum}{}%
}
\section{Annexes}
\subsection*{Etat de l'art : algorithme Glouton \textit{(Exercice 1)}}
\label{algo1Glouton}
\begin{center}
\begin{algorithm}[H]
\textcolor{bleumarine}{\textbf{\caption{\textcolor{bleumarine}{Assemblage\_Glouton()}}}}
\KwEntree{Famille de mots \( F \) de taille \( n \), où \( L \) est la longueur moyenne des mots}
\KwSortie{Chaîne de caractères \( S \) assemblée}
\KwComplexite{Temps : \(\mathcal{O}(n^3 + n^2 L)\), Espace : \(\mathcal{O}(n^2)\)}
\KwDebut \\
    \Indp
     \( S \gets \text{""} \) \textcolor{black!50} ; \( M \gets \) tableau d'entiers de taille \( n \times n \) \textcolor{black!50}{~($\mathcal{O}(n^2)$)}\;
    \KwPour \( i \) allant de \( 0 \) à \( n-1 \) \KwFaire \textcolor{black!50}{~($\mathcal{O}(n^2 L)$)}\\
        \Indp
        \KwPour \( j \) allant de \( 0 \) à \( n-1 \) \KwFaire \textcolor{black!50}{~($\mathcal{O}(n L)$)}\\
            \Indp
            \( M[i][j] \gets \text{longueur du chevauchement maximal entre } F[i] \text{ et } F[j] \) \textcolor{black!50}{~($\mathcal{O}(L)$)}\;
            \Indm
        \KwFinPour \\
        \Indm
    \KwFinPour \\
    \( m \gets 0 \) \textcolor{black!50}\;
    \KwTantQue \( m < n-1 \) \KwFaire \textcolor{black!50}{~($\mathcal{O}(n)$)}\\
        \Indp
        \( \textit{imax} \gets 0 \); \( \textit{jmax} \gets 0 \); \( \textit{max} \gets M[0][0] \) \textcolor{black!50}\;
        \KwPour \( i \) allant de \( 0 \) à \( n-1 \) \KwFaire \textcolor{black!50}{~($\mathcal{O}(n^2)$)}\\
            \Indp
            \KwPour \( j \) allant de \( 0 \) à \( n-1 \) \KwFaire \textcolor{black!50}{~($\mathcal{O}(n)$)}\\
                \Indp
                \KwSi \( \textit{max} < M[i][j] \) \KwAlors \textcolor{black!50}\\
                    \Indp
                    \( \textit{imax} \gets i \); \( \textit{jmax} \gets j \); \( \textit{max} \gets M[i][j] \) \textcolor{black!50}\;
                    \Indm
                \KwFinSi \\
                \Indm
            \KwFinPour \\
            \Indm
        \KwFinPour \\
        \( S \gets S + F[\textit{imax}][0:\text{longueur}(F[\textit{imax}]) - \textit{max}] \) \textcolor{black!50}{~($\mathcal{O}(L)$)}\;
        \( S \gets S + F[\textit{jmax}] \) \textcolor{black!50}{~($\mathcal{O}(L)$)}\;
        \KwPour \( i \) allant de \( 0 \) à \( n-1 \) \KwFaire \textcolor{black!50}{~($\mathcal{O}(n)$)}\\
            \Indp
            \( M[\textit{imax}][i] \gets -1 \); \( M[i][\textit{jmax}] \gets -1 \) \textcolor{black!50}\;
            \Indm
        \KwFinPour \\
        \( m \gets m + 1 \) \textcolor{black!50}\;
        \Indm
    \KwFinTantQue \\
    \KwRetourner \( S \) \textcolor{black!50}\;
\Indm
\KwFin
\end{algorithm}
\end{center}

\subsection*{Etat de l'art : construction du graphe de chevauchement \textit{(Exercice 1)}}

\begin{center}
\begin{algorithm}[H]
\textcolor{bleumarine}{\textbf{\caption{Build\_Graphe\_Chevauchement()}}}
\KwEntree{Famille de mots $\mathcal{F}$ , où chaque \( F_i \) est un mot}
\KwSortie{Graphe de chevauchement \( G = (V, E) \), où \( V \) est l'ensemble des sommets (chaque sommet représente un mot de \( F \)) et \( E \) est l'ensemble des arêtes pondérées par la longueur du chevauchement}
\KwComplexite{Temps : \(\mathcal{O}(n^2 L)\), Espace : \(\mathcal{O}(n^2)\) où \( L \) est la longueur moyenne des mots}
\KwDebut \\
    \Indp
    \( V \gets F \) \;
    \( E \gets "" \) \;
    \KwPour \( i \) allant de \( 0 \) à \( n-1 \) \KwFaire \textcolor{black!50}{~($\mathcal{O}(n^2 L)$)}\\
        \Indp
        \KwPour \( j \) allant de \( 0 \) à \( n-1 \) \KwFaire \textcolor{black!50}{~($\mathcal{O}(n L)$)}\\
            \Indp
            \KwSi \( i \neq j \) \KwAlors \\
                \Indp
                \( k \gets \text{longueur du chevauchement maximal entre } F[i] \text{ et } F[j] \) \textcolor{black!50}{~($\mathcal{O}(L)$)}\;
                \KwSi \( k > 0 \) \KwAlors \\
                    \Indp
                    \( E \gets E \cup \{(F[i], F[j], k)\} \) \;
                    \Indm
                \KwFinSi \\
                \Indm
            \KwFinSi \\
            \Indm
        \KwFinPour \\
        \Indm
    \KwFinPour \\
    \KwRetourner \( G = (V, E) \) \;
\Indm
\KwFin
\end{algorithm}
\end{center}


\begin{center}
\begin{figure}
\begin{algorithm}[H]
\textcolor{bleumarine}{\textbf{\caption{Assemblage\_Graphe\_Chevauchement()}}}
\KwEntree{Famille de mots $\mathcal{F}$; graphe de chevauchement \( G = (V, E) \) construit à partir de $\mathcal{F}$}
\KwSortie{Chaîne de caractères \( S \) assemblée}
\KwComplexite{Temps : \(\mathcal{O}(n^2 L + n^3)\), Espace : \(\mathcal{O}(n^2)\)}
\KwDebut \\
    \Indp
    " Initialisation"\\
    \( S \gets \text{""} \) \;
    \( \text{Visite} \gets \) tableau de booléens de taille \( n \), initialisé à \(\text{FAUX}\) \;
    \( \text{chemin} \gets \) liste vide \;
    \( \text{degré\_entrant} \gets \) tableau d'entiers de taille \( n \), initialisé à \( 0 \) \;

    % Calcul des degrés entrants pour chaque sommet
    \KwPour chaque arête \( (u, v, k) \in E \) \KwFaire \\
        \Indp
        \( \text{degré\_entrant}[v] \gets \text{degré\_entrant}[v] + 1 \) \;
        \Indm
    \KwFinPour

    % Sélection du sommet de départ (celui avec le plus petit degré entrant)
    \( u \gets \) sommet avec le plus petit \(\text{degré\_entrant}\) \;

    % Construction du chemin en suivant les arêtes de poids maximal
    \KwTantQue \( u \neq \text{AUCUN} \) \KwFaire \\
        \Indp
        \( \text{chemin}.ajouter(u) \) \;
        \( \text{Visite}[u] \gets \text{VRAI} \) \;
        % Trouver le sommet non Visite avec le chevauchement maximal
        \( v \gets \) sommet non Visite tel que \( (u, v, k) \in E \) et \( k \) est maximal \;
        \( u \gets v \) \;
        \Indm
    \KwFinTantQue

    % Construction de la superchaîne \( S \) à partir du chemin
    \KwSi \( \text{chemin.taille} > 0 \) \KwAlors \\
        \Indp
        \( S \gets F[\text{chemin}[0]] \) \;
        \KwPour \( i \) allant de \( 1 \) à \( \text{chemin.taille} - 1 \) \KwFaire \\
            \Indp
            \( k \gets \) longueur du chevauchement entre \( F[\text{chemin}[i-1]] \) et \( F[\text{chemin}[i]] \) \;
            \( S \gets S + F[\text{chemin}[i]][k:] \) \;
            \Indm
        \KwFinPour \\
        \Indm
    \KwFinSi

    \KwRetourner \( S \) \;
\Indm
\KwFin
\end{algorithm}
\caption{\underline{Etat de l'art :Algorithme de construction du graphe de chevauchement, comme étudié en cours}}
\end{figure}
\end{center}

\newpage
\subsection*{Spécifications du Système pour les benchmarks}
\begin{table}[h!]
    \centering
    \begin{tabular}{l l}
        \toprule
        \textbf{Catégorie} & \textbf{Détail} \\
        \midrule
        \multicolumn{2}{l}{\textbf{Système}} \\
        Système d'Exploitation & \textbf{Ubuntu 24.04.3 LTS} \\
        Noyau (Kernel) & \textbf{6.14.0-36-generic} \\
        \midrule
        \multicolumn{2}{l}{\textbf{Processeur (CPU)}} \\
        Modèle & \textbf{AMD Ryzen 7 7730U with Radeon Graphics} \\
        Architecture & $\text{x86\_64}$ \\
        Cœurs physiques | Threads & $\text{8 cœurs} \mid \text{16 threads}$ \\
        Fréquence Max. & $4547 \text{ MHz}$ \\
        \midrule
        \multicolumn{2}{l}{\textbf{Mémoire Vive (RAM)}} \\
        Taille Totale & $\text{15 GiB}$ \\
        \midrule
        \multicolumn{2}{l}{\textbf{Graphique (GPU)}} \\
        Contrôleur | Modèle & \textbf{Advanced Micro Devices, Inc. [AMD/ATI]} \\
        Produit (Puce Intégrée) & \textbf{Barcelo (AMD Radeon Graphics)} \\
        \bottomrule
    \end{tabular}
    \label{tab:specs}
    \caption{Spécifications du Système - Ubuntu}
\end{table}


\newpage 
\subsection*{Exemple d'une structure complexe : cas limite de notre algorithme \textit{(Exercice 6 )}}
\begin{figure}[h!]
    \centering
    \includegraphics[width=1\textwidth]{IMAGES/HAAAAAAAAAAAAAA.png}
    \caption{\underline{Visualisation \texttt{Bandage} d'une structure non résolue par l'algorithme.}}
    \label{fig:resolveBubbles_complex}
\end{figure}


\usetikzlibrary{shapes.geometric, arrows, positioning, calc, fit, backgrounds}

\tikzset{
    startstop/.style={rectangle, rounded corners, minimum width=3cm, minimum height=1cm, text centered, draw=black, fill=blue!10},
    process/.style={rectangle, minimum width=3cm, minimum height=1cm, text centered, draw=black, fill=green!10},
    decision/.style={diamond, aspect=2, minimum width=3cm, minimum height=1cm, text centered, draw=black, fill=orange!10},
    arrow/.style={thick,->,>=stealth'},
    block/.style={rectangle, minimum width=3cm, minimum height=1cm, text width=5cm, text centered, draw=black, fill=gray!10},
    substep/.style={rectangle, minimum width=2.5cm, minimum height=0.8cm, text width=5cm, align=center, draw=black!50, fill=white},
    % Nouveaux styles pour les cadres de fond
    phasebox/.style={draw, dashed, rounded corners, inner sep=0.5cm, line width=1pt}
}

\begin{tikzpicture}[node distance=1.2cm, auto, scale=0.7, transform shape]

    % --- NOEUDS ---
    \node (start) [startstop] {Début \texttt{mergeContigs}};
    \node (global_loop_init) [process, below=of start] {Init \texttt{global\_change = true}};
    \node (global_loop_decision) [decision, below=of global_loop_init] {\texttt{global\_change}?};
    \node (end) [startstop, right=of global_loop_decision, xshift=2.5cm] {Fin : Retourner contigs};
    \node (global_loop_reset) [process, below=of global_loop_decision, yshift=-0.5cm] {Reset \texttt{false}};

    % --- PHASE 1 ---
    \node (phase1_label) [block, below=of global_loop_reset] {PHASE 1 : Containment};
    \node (phase1_step1) [substep, below=of phase1_label] {1. Indexation Débuts\\ (start\_map)};
    \node (phase1_step2) [substep, below=of phase1_step1] {2. Scan Inclusion\\ (absorbed)};
    
    % --- PHASE 2 ---
    \node (phase2_label) [block, below=of phase1_step2, yshift=-0.8cm] {PHASE 2 : Extension};
    \node (phase2_step1) [substep, below=of phase2_label] {1. Indexation Graines\\ (seed\_map)};
    \node (phase2_loop_start) [process, below=of phase2_step1] {Pour chaque Maître non absorbé};
    
    \node (try_std_ext) [process, below=of phase2_loop_start] {Tentative 1 : Extension Std};
    \node (std_ext_success) [decision, below=of try_std_ext] {Succès ?};
    \node (merge_success_std) [process, below=of std_ext_success, yshift=-0.5cm] {Change=true, Maître étendu};

    \node (try_rc_ext) [process, right=of std_ext_success, xshift=3cm] {Tentative 2 : Extension RC};
    \node (rc_ext_success) [decision, below=of try_rc_ext] {Succès ?};
    \node (merge_success_rc) [process, below=of rc_ext_success, yshift=-0.5cm] {Change=true, Remplacement RC};

    \node (next_master) [process, below=of merge_success_std, yshift=-1.5cm] {Maître suivant};
    \node (end_loop_phase2) [coordinate, below=of next_master, yshift=-0.5cm] {};

    % --- FLÈCHES ---
    \draw [arrow] (start) -- (global_loop_init);
    \draw [arrow] (global_loop_init) -- (global_loop_decision);
    \draw [arrow] (global_loop_decision) -- node[above] {Non} (end);
    \draw [arrow] (global_loop_decision) -- node[right] {Oui} (global_loop_reset);
    \draw [arrow] (global_loop_reset) -- (phase1_label);
    \draw [arrow] (phase1_label) -- (phase1_step1);
    \draw [arrow] (phase1_step1) -- (phase1_step2);
    \draw [arrow] (phase1_step2) -- (phase2_label);
    \draw [arrow] (phase2_label) -- (phase2_step1);
    \draw [arrow] (phase2_step1) -- (phase2_loop_start);
    \draw [arrow] (phase2_loop_start) -- (try_std_ext);
    \draw [arrow] (try_std_ext) -- (std_ext_success);
    
    \draw [arrow] (std_ext_success) -- node[right] {Oui} (merge_success_std);
    \draw [arrow] (std_ext_success) -- node[above] {Non} (try_rc_ext);
    
    \draw [arrow] (try_rc_ext) -- (rc_ext_success);
    \draw [arrow] (rc_ext_success) -- node[right] {Oui} (merge_success_rc);
    
    % Flèche NON qui contourne large
    \draw [arrow] (rc_ext_success.east) -- node[above] {Non} ++(2,0) |- (next_master.east);

    \draw [arrow] (merge_success_std) -- (next_master);
    \draw [arrow] (merge_success_rc.south) |- (next_master);

    \draw [arrow] (next_master) -- (end_loop_phase2);
    \draw [arrow] (end_loop_phase2) -- ++(-7,0) node[midway, above] {Cycle suivant} |- ([yshift=0.5cm]global_loop_decision.north) -- (global_loop_decision.north);

    % --- CADRES DE FOND (BACKGROUNDS) ---
    % On utilise un layer pour dessiner DERRIÈRE les noeuds existants
    \begin{pgfonlayer}{background}
        
        % Cadre Phase 1 : Englobe le label Phase 1 et l'étape 2
        \node [fit=(phase1_label) (phase1_step2), 
               phasebox, fill=blue!5, draw=blue!30, 
               label={[anchor=north west, inner sep=5pt, color=blue!80]north west:\textbf{PHASE 1}}] {};

        % Cadre Phase 2 : Englobe le label Phase 2, l'extension RC (tout à droite) et le maitre suivant (tout en bas)
        \node [fit=(phase2_label) (try_rc_ext) (merge_success_rc) (next_master) (phase2_loop_start), 
               phasebox, fill=red!5, draw=red!30, 
               label={[anchor=north west, inner sep=5pt, color=red!80]north west:\textbf{PHASE 2}}] {};
               
    \end{pgfonlayer}

\end{tikzpicture}


\begin{table}[h!]
    \centering
        \caption{Répartition du temps d'exécution optimisé de l'Assembleur GraphDBJ.}
    \label{tab:execution_time_new}
    \begin{tabular}{lcc}
        \toprule
        \textbf{Étape} & \textbf{Temps (ms)} & \textbf{Pourcentage du Total} \\
        \midrule
        Création du Graphe & 175 & $53.8\%$ \\
        Fusion des Contigs & 100.5 & $30.9\%$ \\
        Écriture Fichiers & 36 & $11.1\%$ \\
        Génération Contigs & 9.5 & $2.9\%$ \\
        Lecture FASTA & 4.5 & $1.4\%$ \\
        Simplification & 0 & $0.0\%$ \\
        \bottomrule
        \textbf{TOTAL} & $\mathbf{325.5}$ & $\mathbf{100.0\%}$ \\
        \bottomrule
    \end{tabular}
\end{table}

\section{Références}
\vspace{3mm}
\subsection*{Bibliographie}
\printbibliography[heading=none]

\subsection*{Webographie :}
\textbf{QUAST : }
\url{https://github.com/ablab/quast}\\
\textbf{D-GENIES : }\url{https://dgenies.toulouse.inra.fr/}\\
\textbf{Minia : }\url{https://gatb.inria.fr/software/minia/}\\
\textbf{SPades : }\url{https://github.com/ablab/spades}\\

