% !TEX option = -shell-escape
\documentclass[a4paper,11pt]{article}

% --- Langue et encodage ---
\usepackage[french,english]{babel} % Langue du document
\usepackage[T1]{fontenc} % Codage des caractères pour la gestion des accents
\usepackage[utf8]{inputenc} % Encodage UTF-8

% --- Mise en page ---
\usepackage[left=1cm,right=1cm,top=1.7cm,bottom=1.6cm]{geometry} % Gestion des marges
\usepackage{setspace} % Gestion des espaces interlignes
\usepackage{titlesec} % Personnalisation des titres de sections
\usepackage{titling} % Personnalisation du titre
\usepackage{fancyhdr} % Personnalisation des en-têtes et pieds de page
\usepackage{paracol} % Diviser la page en colonnes
\usepackage[bottom]{footmisc} % note bas de page
\addtolength{\skip\footins}{20pt}  % Ajuste la valeur selon tes besoins

% --- Mathématiques ---
\usepackage{mathtools} % Paquet pour des équations et symboles mathématiques
\usepackage{caption}
%--Graphiques 3 dimensions --
\usepackage{pgfplots}
\pgfplotsset{compat=1.18}
\usepackage{pgfplotstable}

\usepackage{siunitx} % Écriture avec la notation scientifique

% --- Flottants et images ---
\usepackage[section]{placeins} % Forcer les flottants à rester dans les sections
\usepackage{graphicx} % Insérer des images
\usepackage[justification=centering]{caption} % Légendes centralisées
\usepackage{subcaption} % Sous-légendes pour des figures
% Redéfinition du style des légendes
\DeclareCaptionLabelFormat{blue}{\textcolor{bleumarine}{#1~#2:}}
\captionsetup{
    labelfont={color=bleumarine,bf}, % Couleur et gras pour le texte
    labelformat=blue % Format défini pour les légendes
}
\usepackage{wallpaper} % Utiliser des fonds ou des images de fond
\usepackage{wrapfig} % Mise en page personnalisée pour les images

% --- Tableaux ---
\usepackage{array,multirow,makecell} % Personnalisation avancée des tableaux
\usepackage{booktabs} % Amélioration des tables (lignes fines, moyennes, épaisses)
\usepackage{adjustbox}
\usepackage{colortbl} % Coloration des cellules de tableaux
\usepackage{xcolor} % Couleur des tableaux   
\usepackage{diagbox} % Pour créer des cellules en diagonale
\usepackage{longtable} % Tables sur plusieurs pages

% --- Code source ---
\usepackage{minted} % Mise en forme de code source avec coloration syntaxique
\usepackage{listings} % Insertion et mise en forme du code source
\usepackage{everypage}

% --- Bibliographie et références ---
\usepackage[backend=biber]{biblatex} % Gestion de la bibliographie
\addbibresource{reference.bib}
\usepackage{csquotes} % Gestion des guillemets typographiques
\usepackage{hyperref} % Liens hypertexte sans surlignage
\usepackage{cleveref} % Références intelligentes

% --- Textes et acronymes ---
\usepackage{lipsum} % Générer du texte de remplissage
\usepackage{acro} % Gestion des acronymes
\usepackage{nomencl} % Création d'une nomenclature (glossaire)
\makenomenclature % Commande pour générer la nomenclature

% --- Dessins et graphiques ---
\usepackage{tikz} % Nécessaire pour smartdiagram
\usetikzlibrary{shapes.geometric, arrows.meta, calc, positioning,shadows}
\usepackage{eso-pic} % Gestion d'images en arrière-plan
\usepackage{tcolorbox} % Création de boîtes de couleur pour des équations ou du texte
\usepackage{smartdiagram} % Créer les figures simplement
\usesmartdiagramlibrary{additions} %Dépendances smartdiagram
\usepackage{graphicx} % Package nécessaire pour inclure des images
\usetikzlibrary{trees}

% --- Divers ---
\usepackage{appendix} % Gérer les annexes
\usepackage{multicol} % Texte en colonnes
\usepackage{lastpage} % Référence à la dernière page
\usepackage{ragged2e} % Justification d'un texte à l'intérieur d'un parbox
\usepackage{xcolor}
\usepackage{algorithm2e}
\usepackage{amsmath, amssymb}
\usepackage{setspace}
\usepackage{url}
% --- Couleurs ---
\definecolor{bleumarine}{RGB}{20,40,160}
\definecolor{red}{RGB}{139,0,0}
\definecolor{vert}{RGB}{0,100,0}
\definecolor{gris}{RGB}{100,100,100}

% Formatage des section

\titleformat{\section}[block]
  {\normalfont\Large\bfseries\rmfamily\color{bleumarine}} % Police, taille, graisse et couleur
  {\thesection}{1em}{}[\titlerule] % Numérotation et ligne sous le titre
\titlespacing{\section}{5pt}{10pt}{10pt} % Espacement avant et après
\renewcommand{\labelitemi}{\textemdash}
% Formatage des sous-sections
\titleformat{\subsection}[block]
  {\normalfont\large\bfseries\rmfamily\color{bleumarine}} % Police, taille, graisse et couleur
  {\thesubsection}{1em}{} % Numérotation
\titlespacing{\subsection}{5pt}{8pt}{5pt} % Espacement avant et après

% Formatage des sous-sous-sections
\titleformat{\subsubsection}[block]
  {\normalfont\normalsize\bfseries\rmfamily\color{bleumarine}} % Police, taille, graisse et couleur
  {\thesubsubsection}{1em}{} % Numérotation
\titlespacing{\subsubsection}{5pt}{8pt}{5pt} % Espacement avant et après

% Redéfinir ma commande \bibliography pour personnaliser les liens
\usepackage{etoolbox}
\pretocmd{\bibliography}{%
  \hypersetup{
    colorlinks=true,  % Active les liens colorés
    linkcolor=black,   % Lien dans le texte
    urlcolor=black,    % Lien URL dans la bibliographie
    filecolor=black    % Lien vers les fichiers
  }%
}{}{}
\hypersetup{
    citecolor=darkblue,  % Couleur des références dans le texte
}
%---------------------------Diagramme de conclusion ----------------------------------------
% Définition de la mise en forme listing (pour l'intégration des codes et pseudoCode)
\definecolor{background}{HTML}{F1F1F0}
\definecolor{string}{HTML}{DD1144}
\definecolor{keyword}{HTML}{0077AA}
\definecolor{comment}{HTML}{44AA99}
\definecolor{identifier}{HTML}{000000}
\definecolor{framecolor}{HTML}{CCCCCC}  

\lstset{
  backgroundcolor=\color{background}, % Couleur de fond
  basicstyle=\small\ttfamily, % Taille et police par défaut
  keywordstyle=\color{blue}, % Mots-clés en bleu
  commentstyle=\color{comment}, % Commentaires
  stringstyle=\color{string}, % Chaînes de caractères
  identifierstyle=\color{identifier}, % Identifiants
  numbers=left, % Numérotation des lignes
  numberstyle=\tiny\color{gray}, % Style de numérotation
  language=bash, % Langage utilisé (bash)
  frame=single, % Encadrement du code
  rulecolor=\color{framecolor}, % Couleur du cadre
  framesep=2pt, % Espacement autour du code
  framerule=0.5pt, % Épaisseur du cadre
  breaklines=true, % Retour à la ligne si le code est trop long
  morekeywords={grep, cut, tr, efetch, nano, build, touch, apptainer, run, shell, mkdir, mv, ln, snakemake,git,clone,tree}, % Ajouter && comme un mot-clé
  emph={\&\&}, % Emphases supplémentaires
  emphstyle=\color{red}, % Met en rouge les opérateurs &&
  inputencoding=utf8, % Encodage UTF-8 pour les caractères spéciaux
  extendedchars=true, % Permet l'utilisation des caractères spéciaux
  showstringspaces=false,
  literate=%
    {é}{{\'e}}1
    {è}{{\`e}}1
    {ê}{{\^e}}1
    {ë}{{\¨e}}1
    {à}{{\`a}}1
    {ç}{{\c{c}}}1
    {ù}{{\`u}}1
    {û}{{\^u}}1
    {î}{{\^i}}1
    {ï}{{\¨i}}1
    {ô}{{\^o}}1
}

\lstdefinestyle{cpp}{
  language=C++,
  morekeywords={struct, uint64_t, uint32_t, std, vector, bool, nullptr, size_t},
  keywordstyle=\color{blue!80!black},
  commentstyle=\color{green!60!black},
}

\usepackage{amsmath, amssymb}
%_________________________ Définition de la bibliographie ____________________
\title{Titre} %Titre du fichier

% Définir un nouvel environnement de flottant 
\newfloat{listing}{htbp}{lop}
\floatname{listing}{Code}
\newenvironment{lexique}{\begin{itemize}[itemsep=-5pt, left=0pt]}{\end{itemize}}

%__________Montage de la table des abbréviation___________

\setlength{\arrayrulewidth}{0.5mm}
\setlength{\columnsep}{0.9cm}
\renewcommand{\arraystretch}{0.9} % Ajuste l'espacement des lignes
%_________________________ Montage du contour ____________________

\AtBeginShipout{
    \begin{tikzpicture}[remember picture, overlay]
        \draw [line width=0.1mm,lightgray] 
            ($(current page.north west) + (0.5cm,-0.5cm)$) 
            rectangle 
            ($(current page.south east) + (-0.5cm,0.5cm)$);
    \end{tikzpicture}
}
%-------------------------------------------------------------------------------------------------
%                              DEFINITIONS DU GLOSSAIRE 
%-------------------------------------------------------------------------------------------------
%________________________Définition de la mise en forme  des pieds de pages et en-tête ___________________
%Définir un fond d'image pour les pages sauf la première
\newcommand{\fairemarges}{
    \makenomenclature
    \pagestyle{fancy}
    \fancyhf{} % Réinitialise les en-têtes et pieds de page

    \setlength{\footskip}{-35pt} % Hauteur du pied de page
    \setlength{\headheight}{10pt} % Hauteur de l'en-tête
    \setlength{\headsep}{0pt} % Espacement entre l'en-tête et le texte 

    % Pied de page gauche : seulement le numéro de page
    \fancyfoot[L]{%
        {\sffamily \Large \textbf{\thepage}}%
    }
    
    % Pied de page droit : titre de section (optionnel)
    \fancyfoot[R]{%
        {\sffamily \normalsize \textit{\nouppercase{\leftmark}}}%
    }

    \renewcommand{\headrulewidth}{0pt}
    \renewcommand{\footrulewidth}{0.3pt}
    % Couleur des lignes
    \renewcommand{\footrule}{{\color{bleumarine}\hrule width\headwidth height\footrulewidth \vskip-\footrulewidth}}
    \renewcommand{\headrule}{{\color{bleumarine}\hrule width\headwidth height\headrulewidth \vskip-\headrulewidth}}
}

\renewcommand{\headrulewidth}{0pt}
\renewcommand{\footrulewidth}{0.3pt}
    % Couleur des lignes
    \renewcommand{\footrule}{{\color{bleumarine}\hrule width\headwidth height\footrulewidth \vskip-\footrulewidth}}
    \renewcommand{\headrule}{{\color{bleumarine}\hrule width\headwidth height\headrulewidth \vskip-\headrulewidth}}
\usepackage{tocloft}  
    
\renewcommand{\contentsname}{Table des matières}  

\renewcommand{\cftsecfont}{\normalfont\small\bfseries\color{bleumarine}}
\renewcommand{\cftsubsecfont}{\normalfont\small\color{black}}
\renewcommand{\cftsecpagefont}{\small\color{black}}
\renewcommand{\cftsubsecpagefont}{\small\color{black}}
\setlength{\cftbeforetoctitleskip}{0pt}
\setlength{\cftaftertoctitleskip}{5pt}
\setlength{\cftbeforesecskip}{2pt}
\setlength{\cftbeforesubsecskip}{1pt}
\setcounter{tocdepth}{2}

\newcommand{\HideTOCPageNumbers}{%
    % Supprime le numéro de page dans le TOC pour les sections et sous-sections
    \renewcommand{\cftsecpagefont}{}     
    \renewcommand{\cftsubsecpagefont}{}  
    \renewcommand{\cftsecafterpnum}{}    
    \renewcommand{\cftsubsecafterpnum}{} 
}
 
\fairemarges % Applique les marges et configurations

\usepackage{algorithm2e}
\usepackage{lmodern}
\usepackage{setspace}

% --- Configuration algorithm2e ---
\SetKwInput{KwEntree}{\textcolor{bleumarine}{Entrée}}
\SetKwInput{KwSortie}{\textcolor{bleumarine}{Sortie}}
\SetKwInput{KwComplexite}{\textcolor{vert}{\textbf{Complexité}}}
\SetKw{KwDebut}{Début}
\SetKw{KwFin}{Fin}
\SetKw{KwSi}{Si}
\SetKw{KwFinSi}{FinSi}
\SetKw{KwSinon}{Sinon}
\SetKw{KwAlors}{Alors}
\SetKw{KwPour}{Pour}
\SetKw{KwFinPour}{FinPour}
\SetKw{KwTantQue}{TantQue}
\SetKw{KwFinTantQue}{FinTantQue}
\SetKw{KwTantQue}{TantQue}
\SetKw{KwRetourner}{\textcolor{red}{Retourner}}
\SetKw{KwReinitialiser}{\textcolor{red}{Reinitaliser}}
\SetKw{KwContinuer}{\textcolor{red}{Continuer}}
\SetKw{KwSupprimer}{\textcolor{red}{Supprimer}}
\SetKw{KwReserver}{\textcolor{red}{Reserver}} 
\SetKw{KwAfficher}{Afficher}
\SetKw{KwFaire}{Faire}
\SetKw{KwSortir}{\textcolor{red}{Sortir}}
\SetKw{KwOuvrir}{Ouvrir}
\SetKw{KwFermer}{\textcolor{red}{Fermer}}
\SetKw{KwAjouter}{\textcolor{red}{Ajouter}}
\SetKw{KwChoisir}{\textcolor{red}{Choisir}}
\SetKw{KwDans}{Dans}
\SetKwProg{Fn}{Fonction}{}{}
\SetKw{KwAllant}{allant de}
\RestyleAlgo{ruled}
\SetAlgoNlRelativeSize{-1}
\DontPrintSemicolon
\onehalfspacing

\begin{document}
\input{Partie_1_Page_de_garde}
\vspace{-12mm}
\section{Introduction}
\subsection{Stratégies étudiées en cours (Exercice~1)}

Nous avons vu que la recherche d’une \underline{superchaîne optimale} contenant tous les mots d’une famille \(\mathcal{F}\) est un problème NP-difficile. Par conséquent, nous utilisons des \underline{approches heuristiques} : celles-ci fournissent une solution, mais sans garantie d’optimalité. Trois méthodes ont été étudiées et développées durant le semestre. Comme demandé, voici une synthèse de chacune.

\subsubsection{Méthode gloutonne}

Cette première approche repose sur les \underline{choix locaux optimaux} (maximisation des \textit{overlaps}), mais sans assurance d’optimalité globale. On va comparer nos mots deux à deux pour identifier leurs chevauchements. Sélectionner le plus long et fusionner les deux mots concernés (créer des contigs). On répéte ainsi l’opération jusqu’à obtenir la superchaîne $\mathcal{S}$. On stock nos \textit{overlaps} dans une matrice $\mathcal{M}$ de $n x n $ séquences. Ci-dessous la stratégie construite à partir de la correction du TD :  \\

\begin{center}
\begin{algorithm}[H]
\textcolor{bleumarine}{\textbf{\caption{\textcolor{bleumarine}{Assemblage\_Glouton()}}}}
\KwEntree{Famille de mots \( F \) de taille \( n \), où \( L \) est la longueur moyenne des mots}
\KwSortie{Chaîne de caractères \( S \) assemblée}
\KwComplexite{Temps : \(\mathcal{O}(n^3 + n^2 L)\), Espace : \(\mathcal{O}(n^2)\)}
\KwDebut \\
    \Indp
     \( S \gets \text{""} \) \textcolor{black!50} ; \( M \gets \) tableau d'entiers de taille \( n \times n \) \textcolor{black!50}{~($\mathcal{O}(n^2)$)}\;
    \KwPour \( i \) allant de \( 0 \) à \( n-1 \) \KwFaire \textcolor{black!50}{~($\mathcal{O}(n^2 L)$)}\\
        \Indp
        \KwPour \( j \) allant de \( 0 \) à \( n-1 \) \KwFaire \textcolor{black!50}{~($\mathcal{O}(n L)$)}\\
            \Indp
            \( M[i][j] \gets \text{longueur du chevauchement maximal entre } F[i] \text{ et } F[j] \) \textcolor{black!50}{~($\mathcal{O}(L)$)}\;
            \Indm
        \KwFinPour \\
        \Indm
    \KwFinPour \\
    \( m \gets 0 \) \textcolor{black!50}\;
    \KwTantQue \( m < n-1 \) \KwFaire \textcolor{black!50}{~($\mathcal{O}(n)$)}\\
        \Indp
        \( \textit{imax} \gets 0 \); \( \textit{jmax} \gets 0 \); \( \textit{max} \gets M[0][0] \) \textcolor{black!50}\;
        \KwPour \( i \) allant de \( 0 \) à \( n-1 \) \KwFaire \textcolor{black!50}{~($\mathcal{O}(n^2)$)}\\
            \Indp
            \KwPour \( j \) allant de \( 0 \) à \( n-1 \) \KwFaire \textcolor{black!50}{~($\mathcal{O}(n)$)}\\
                \Indp
                \KwSi \( \textit{max} < M[i][j] \) \KwAlors \textcolor{black!50}\\
                    \Indp
                    \( \textit{imax} \gets i \); \( \textit{jmax} \gets j \); \( \textit{max} \gets M[i][j] \) \textcolor{black!50}\;
                    \Indm
                \KwFinSi \\
                \Indm
            \KwFinPour \\
            \Indm
        \KwFinPour \\
        \( S \gets S + F[\textit{imax}][0:\text{longueur}(F[\textit{imax}]) - \textit{max}] \) \textcolor{black!50}{~($\mathcal{O}(L)$)}\;
        \( S \gets S + F[\textit{jmax}] \) \textcolor{black!50}{~($\mathcal{O}(L)$)}\;
        \KwPour \( i \) allant de \( 0 \) à \( n-1 \) \KwFaire \textcolor{black!50}{~($\mathcal{O}(n)$)}\\
            \Indp
            \( M[\textit{imax}][i] \gets -1 \); \( M[i][\textit{jmax}] \gets -1 \) \textcolor{black!50}\;
            \Indm
        \KwFinPour \\
        \( m \gets m + 1 \) \textcolor{black!50}\;
        \Indm
    \KwFinTantQue \\
    \KwRetourner \( S \) \textcolor{black!50}\;
\Indm
\KwFin
\end{algorithm}
\end{center}

\subsection{Notre approche}

\subsection{Choix du langage \textit{(Exercice 3)}}
Dans la création de cette assembleur, la question de langage c'est posé.

Nous avons le choix entre pléthore de langages, notamment \texttt{Python}, \texttt{C++} ou d'autre encore comme le go.
Notre choix finale se porte sur le \texttt{C++}, prédécesseur du \texttt{C}, qui est un langage compilé donnant des performances excellentes en laissant le développeur contrôlé chaque point du programme. 
Il permet de définir et libérer la mémoire manuellement, permettant un gain de place et de temps important. 

D'autre langages comme \texttt{Python} aurait peut être facilité la programmation et augmenté la lisibilité et la réutilisation par tous, mais a contrepartie une perte de performance.

\newpage

\subsubsection{La classe \texttt{graphdbj.cpp}}
\textbf{\underline {La construction (dans les grandes lignes) :}}\\
Dans cette partie, nous revenons sur les choix conceptuels de notre assembleur, l’objectif étant de répondre, nous l’espérons, au plus près des attendus du TP. Les différentes classes décrites s’articulent autour de la classe \texttt{GraphDBJ}, qui constitue le coeur du programme : création du graphe, simplification et recherche de chemin(s). Le constructeur du \emph{graphe de De Bruijn}, dans sa version synthétique :

\begin{algorithm}[H]
\textcolor{bleumarine}{\textbf{\caption{\textcolor{bleumarine}{constructGraph(converter, k)}}}}
\KwEntree{Objet \texttt{Convert} contenant les lectures ; taille des k-mers \texttt{k}}
\KwSortie{Graphe de De Bruijn avec noeuds et arêtes}
\KwComplexite{Temps : $\mathcal{O}(N \cdot k)$, où $N$ est la somme des longueurs des lectures}
\KwDebut
\Indp\\
    \KwSi $2 \leq k \leq 32$ \KwSinon \\
    \KwRetourner Erreur \textcolor{black!50}{~($\mathcal{O}(1)$)}\;

    bv $\gets$ \texttt{converter}.getBitVector() \;
    read\_ends $\gets$ \texttt{converter}.getEndPos() \;
    current\_read\_start $\gets 0$ \;

    \KwPour end\_pos \KwDans read\_ends \\
    \Indp
        read\_len $\gets (end\_pos - current\_read\_start)/2$ \;
        \KwSi{read\_len $\ge k$} \KwAlors \\
        \Indp
            \KwPour i = 0 \KwAllant read\_len - k \\
            \Indp
                u\_fwd $\gets$ \texttt{extractKmerValue}(bv, current\_read\_start + 2*i, k-1) \;
                v\_fwd $\gets$ \texttt{extractKmerValue}(bv, current\_read\_start + 2*i + 2, k-1) \;
                u\_rev $\gets$ \texttt{getReverseComplement}(u\_fwd, k-1) \;
                v\_rev $\gets$ \texttt{getReverseComplement}(v\_fwd, k-1) \;

                \KwAjouter \texttt{arête}(u\_fwd $\to$ v\_fwd) , \texttt{arête}(v\_rev $\to$ u\_rev)   \textcolor{black!50}{~($\mathcal{O}(1)$ amorti)}\;
            \Indm
        \Indm
        current\_read\_start $\gets$ end\_pos \;
    \Indm
\KwFin
\end{algorithm}

Le choix de représenter les noeuds comme des $(k-1)$-mers et les arêtes comme des transitions de taille $k+1$ s’appuie sur la formalisation du graphe étudiée en cours. Cette approche facilite également la détection d’erreurs structurées (tips, bulles), contrairement à l’approche par \textit{overlap} (cf. annexe), plus coûteuse. Dans l’algorithme  \texttt{constructGraph}, on effectue une conversion des k-mers en valeurs entières via \texttt{extractKmerValue()}. Cela a été l’une des premières difficultés rencontrées, notamment pour éviter les « débordements » et garantir la cohérence entre les brins \textit{forward} et \textit{reverse}. Dans la deuxième structure itérative, on génère simultanément le complément inverse en ajoutant les deux arêtes, comme discuté lors de la dernière séance de TP. Cela permet d’obtenir un graphe navigable quel que soit le sens de lecture. Cet aspect est fondamental pour de vrais jeux de données. Ensuite, les noeuds et les arêtes sont exportés en GFA, permettant de représenter le graphe de contigs assemblés avec leur couverture (que nous exploitons dans les outils auxiliaires nous le verrons). Ces explications générales étant posées, nous vous proposons de rdétailler dans la suite de cette section certains aspects qui nous paraissent pertinents. \\

\textbf{\underline{Simplification du Graphe :}}

En générant nos contigs, nous nous sommes aperçus d’un certain nombre de problèmes à gérer pour obtenir des résultats réalistes, tant en quantité qu’en qualité des contigs. Dans ce sens, nous avons implémenté plusieurs méthodes pour nettoyer le graphe afin d’éliminer les éléments qui nous paraissaient, \textit{a priori}, artefactuels. Cette étape (appliquée dans le \texttt{main}) est orchestrée séquentiellement par deux fonctions de la classe \texttt{GraphDBJ}, jusqu’à stabilisation ou jusqu’à l’atteinte du nombre maximal de passes \texttt{max\_passes\_pop}.\\

\begin{wrapfigure}{r}{0.50\textwidth}
    \centering
    \includegraphics[width=0.6\linewidth]{IMAGES/Tip.png}
    \caption{\underline{Illustration d'un tip dans le graphe d'assemblage.}}
    \label{fig:clipTips}
\end{wrapfigure}

Le premier problème qui nous parait essentiel à aborder est la \textbf{suppression des \textit{Tips}}. Cela revient à dire : exclure les chemins terminaux courts dans le graphe. Ils sont générés, d'après la littérature, majoritairement par des erreurs de séquençage. Pour chaque noeud terminal \(v \in V\), on remonte le chemin \(\mathcal{T} = (v_1, v_2, \dots, v_n)\) jusqu'au noeud d'ancrage \(a\), où se produit une bifurcation ou jusqu'à un un premier seuil de longueur matérialiser dans le code par \(n \leq \texttt{TOPO\_MAX\_LEN}\). Les \textit{tips} de cette taille seront alors considéré comme des erreur et déconnectés. Viens ensuite une phase de \og rédemption \fg  ou les \textit{tips} en dessous du seuil \(n \leq \texttt{RCTC\_MAX\_LEN}\), pour \textbf{Ratio de Couverture de Tip-To-Core}~(RCTC) et avec une couverture moyenne inférieure à celle du k-mer d'ancrage sont déconnecté. Cette approche élégante est tirée de l'algorithme de \texttt{minia}\textsuperscript{\cite{chikhi_space-efficient_2013}}, nous avons décidé de l'implémenter dans notre algorithme. Plus formellement, on calcule alors la \og couverture moyenne du tip \fg, pour ensuite la comparer avec la couverture du nœud d'ancrage \(\text{Coverage}(a)\). Nous avons choisi de supprimé le tip si \textsuperscript{\cite{chikhi_space-efficient_2013}}:  
\[
 n \le \texttt{TOPO\_MAX\_LEN}  \texttt{~et~}   (n \le \texttt{RTC\_MAX\_LEN} \quad \&\& \quad \text{Coverage}(a) > \bar{c}_T) ~ | ~  \bar{c}_T = \frac{1}{n} \sum_{i=1}^{n} \text{Coverage}(v_i)
\]  
En pratique, on utilise \texttt{disconnectNodes(a, v)} pour couper la branche et marque tous les noeuds du tip comme \texttt{should\_removed = true}. Ce choix permet de retirer les chemins artefactuels tout en préservant les structures linéaires réelles, plus longues et mieux couvertes. Le second problème rencontré dans les graphes de \textit{De Bruijn} est celui des bulles.\\
\begin{wrapfigure}{l}{0.45\textwidth}
\vspace{-8mm}
    \centering
    \includegraphics[width=0.6\linewidth]{IMAGES/bulle.png}
    \caption{\underline{Illustration d'une bulle avec \texttt{Bandage}}}
    \label{fig:resolveBubbles_simple}
\end{wrapfigure}

 Elles vont apparaître lorsqu'un nœud source $s$ se divise en deux chemins parallèles avant reconverger vers un nœud d'arrivée $t$ . Le motif résultant dans sa grade majorité à des variations biologiques et/ou erreur de séquençage. Si elles ne sont pas résolue elles entraînent une fragmentation artificielle des \textit{contigs}. D'un point de vue formel, pour notre $G = (V, E)$, une bulle est définie par deux chemins  :
Soient \(s,t\in V\). Considérons l'ensemble \(\mathcal{P}(s,t)\) des chemins simples allant de \(s\) à \(t\).
Nous nous intéressons à deux éléments distincts \(\mathcal{P}_1,\mathcal{P}_2\in\mathcal{P}(s,t)\) vérifiant $ \mathcal{P}_1\neq\mathcal{P}_2 \text{~et~} \mathcal{P}_1\cap\mathcal{P}_2=\{s,t\}.$ 
\clearpage 
On dois définir également la couverture cumulée $\mathcal{C}$ du chemin $\mathcal{P}$, on aura ainsi une résolution de la bulle en sélectionnant $\mathcal{P}_{\max}$ et à supprimer l'autre chemin : \\ 
\[
C(\mathcal{P}) = \sum_{v \in \mathcal{P}} \mathrm{Coverage}_{v}\; ;\quad
\mathcal{P}_{\max} = \arg\max_{\mathcal{P} \in \{\mathcal{P}_1,\mathcal{P}_2\}} C(\mathcal{P})\; ;\quad
\{\mathcal{P}_1,\mathcal{P}_2\} \setminus \{\mathcal{P}_{\max}\}
\]

La méthode \texttt{resolveBubbles()} fait ce travail de détection pour chacun des noeuds possèdants deux voisins qui finissent par converger vers un noeud commun et la profondeur $\mathcal{P}$est  limitée par \texttt{SEARCH\_DEPTH\_FACTOR}). L'algorithme compare $\mathcal{C}$ de $\mathcal{P}_{1} \text{et} \mathcal{P}_{2}$, la branche la moins couverte est marquée à travers  \texttt{removed = true}, puis \texttt{disconnectNodes()}. \\

\textbf{\underline{Génération des Contigs Bruts :}}\\

Après avoir construit notre graphe et effectué les étapes de simplification (\textit{Tip, bulles},suppression de nos chemins artefactuels), il nous reste à générer nos contigs, i.e les chemins linéaires maximaux $P$ ou aucune \og ambiguité topologique \fg n'est présente. Formellement pour tout noeuds intermédiaires $i$, de degrès entrant $ \deg^{-}$  et sortant $\deg^{+}$  :
\[
\mathcal{P} = (v_1, v_2, \dots, v_m) \mid  \forall\, i \in \{2,\dots,m-1\},\; 
\deg^{-}(v_i) = \deg^{+}(v_i) = 1.
\]

Finalement si on se rapporte à ce que l'on à vu en cours, on est en droit de dire que ces chemins sont des zones du génome ou l'assemblage est déterministe. Le challenge algorithmique se réparti en trois aspects : détecter les bons points de départ, prolonger le chemin tant que la structure reste univoque et enfin reconstruire la séquence nucléotidique à partir de nos $(k-1)$-mers encodés dans les noeuds de $G$. Pour construire cela on considère que chaque noeud \(v\) est un point de départ du contig si :  $\deg^{-}(v) \neq 1 \quad \text{ou} \quad \deg^{+}(v) \neq 1$ en imposant \(\deg^{+}(v) \ge 1\). À partir d’un tel nœud \(v_0\), on étend le 
chemin tant que : $\deg^{-}(v_i)=\deg^{+}(v_i)=1.$ Le contig sera maximal quand on atteint un noeud \(v_m\) tel que \(\deg^{+}(v_m)\neq 1\). Considérant ces règles, il nous faut reconstruire la séquence en concaténant le $(k-1)$mer du premier noeud, puis à chaque arc \((v_i \to v_{i+1})\), le dernier nucléotide du $(k-1)$-mer de \(v_{i+1}\). Ainsi, pour un chemin de longueur \(m\), la longueur du contig est logiquement : $(k-1) + (m-1).$\\

L'ensemble de cette logique est implémentée dans la méthode \texttt{generateContigs()}, dont nous vous proposons l'algorithme avec uniquement les grandes étapes de la fonction, à la page suivante. Il parcourt l'ensemble des noeuds du graphe simplifié (plus de \textit{tips,bubbles}), pour déterminer ceux qui peuvent initier un contig. Pour chaque noeud candidat \texttt{startNode}, on vérifie qu'il n’a pas déjà été visité (\texttt{visited}), pour éviter des contigs dupliqués. On test ensuite notre condition topologique  $\deg^{-}(v) \neq 1 \quad \text{ou} \quad \deg^{+}(v) \neq 1$, ce qui va correspondre à soit un noeud de départ potentiel, soit un noeud sans parent, soit une bifurcation (réelle car restante).
Ensuite on initialise notre contig avec le $(k-1)$-mer du nœud de départ avec (\texttt{addKmerToBitVector()}). Puis on fait un parcours linéaire du graphe en suivant les arêtes sortantes tant que chaque noeud intermédiaire satisfait un chemin simple, $\deg^{-} = \deg^{+} = 1$. En cas de bifurcation, on à fait un choix heuristique basé sur la couverture (\texttt{COVERAGE\_RATIO}), en s'inspirant du fonctionnement de \texttt{minia} et de notre stratégie de résolution des \textit{bubbles}, pour prolonger le contig uniquement si un chemin dominant se dégage. Enfin, La reconstruction de la séquence repose sur le codage des $(k-1)$-mers,fonction \texttt{kmerToString()}. Lors du parcours, seul le dernier nucléotide du nœud suivant est ajouté au contig, conformément à la définition du graphe de De Bruijn vu en cous et à la logique d’extension décrite dans le pseudocode. Cette stratégie évite toute redondance ( \textit{à priori}). Pour finir on encapsule notre contig dans une structure qui contient: la séquence complète, la couverture moyenne du chemin, la longueur totale et un identifiant unique:\\
\begin{algorithm}[H]
\textcolor{bleumarine}{\textbf{\caption{\textcolor{bleumarine}{generateContigs() (Simplifié)}}}}
\KwEntree{Graphe de $G=(V,E)$, $k$, COVERAGE\_RATIO, MAX\_CONTIG\_LEN}
\KwSortie{Liste de contigs en BitVector.}
\KwComplexite{Temps : $\mathcal{O}(|V| + \sum_{v \in V} \deg^+(v))$}
\KwDebut\\
\Indp
    \texttt{contigs} $\gets$ liste vide \;
    \texttt{visited} $\gets$ Dictionnaire pour tous les noeuds \;

    \KwPour chaque \texttt{startNode} \KwDans $V$ \\
    \Indp
        \KwSi{\texttt{startNode.removed} ou \texttt{visited[startNode]}} \KwAlors \\
        \Indp\KwContinuer\\ \Indm
        \KwFinSi\\
        \KwSi{\texttt{startNode.parents.empty()} ou \texttt{startNode.parents.size() > 1}} \KwAlors\\
        \Indp
            \texttt{currentContig} $\gets$ BitVector vide \;
            addKmerToBitVector(\texttt{currentContig}, startNode.p, k-1) \;
            \texttt{visited[startNode]} $\gets$ true \;
            \texttt{curr} $\gets$ startNode \;

            \KwTantQue{sanity\_check < MAX\_CONTIG\_LEN}\\
            \Indp
                \texttt{next} $\gets$ \KwChoisir enfant unique ou (dominant selon COVERAGE\_RATIO)\\
                \texttt{[...]}\\
                \KwSi{\texttt{next} = null ou \texttt{next.removed} ou \texttt{visited[next]}} \KwAlors\\
                \Indp\KwSortir\\
                \Indm
                \KwFinSi \\
                \KwAjouter Dernier nucléotide de \texttt{next} à \texttt{currentContig} \;
                \texttt{visited[next]} $\gets$ true \;
                \texttt{curr} $\gets$ \texttt{next} \;
            \Indm   
            \KwAjouter \texttt{currentContig} à {contigs} \;
            \KwFinTantQue\\\
            \Indm \KwFinSi\\
\Indm
\KwFinPour\\
\Indm \KwFin
\end{algorithm}
\vspace{5mm}
\textbf{\underline{Fusion des contigs :}}\\

Nous avons constatés que malgré tout ce travail de simplification, les contigs obtenus avec \texttt{generateContigs()}, restent très \og fragmentés \fg. A l'aide de \textit{bandage} et après discussions, nous pensons que ce problème vient en grande partie d'une orientation inversé de certains contigs (\textit{reverse complément}). Mais aussi à des chevauchements partiels non détectés lorsque nous parcourons le graphe. 
La fonction \texttt{mergeContigs()} sert à \og recoudre \fg ces derniers pour produire  des séquences plus longues et cohérentes en prenant garde que l'on ne perde pas d'information. Simplement, on peut distinguer deux grandes phases. La phase d'inclusion (\textit{Containment}) qui élimine les contigs totalements inclus dans d'autres, et la phase d'extension (\textit{seed \& extend}) qui cherche à fusionner les contigs restants en fusionnant les chevauchements internes et en gérant l'orientation. Soit notre ensemble de contigs $C$ tel que : 
\(\mathcal{C} = \{C_1, \dots, C_n\}\) et \(k\) la taille du k-mer utilisé pour l’indexation.\\[1ex]
Dans la phase d'inclusion (\textit{Containment}), chaque contig \(C_j\) est comparé à tous les autres contigs \(C_i\) (\(i \neq j\)) pour détecter s'il est entièrement contenu dans un contig plus long, on marque \(C_j\) comme absorbé si :  
\[
\text{absorbed}[j] = \text{true} \iff \exists i \neq j,\, C_j \subseteq C_i \text{ (avec une tolérance d'erreur } \epsilon_\text{contain}\text{)}
\]  
Ce qui est intéressant ici, c'est que tous les nucléotides de \(C_j\) correspondent, à une petite marge d'erreur près, à une sous-séquence d'un autre contig \(C_i\). Dans ce cas que, \(C_j\) est considéré redondant et n'est plus conservé pour la phase suivante. Dans la phase d'extension (\textit{Seed \& Extend}), pour chaque contig maître \(C_m\), on recherche un contig candidat \(C_c\) possédant un k-mer \emph{seed} \(s\) qui correspond à une sous-séquence de \(C_m\) :  
\[
\exists s \in C_c : s = C_m[\text{pos}:\text{pos}+k-1] \implies 
C_m \gets C_m[0:\text{align\_start}-1] \oplus C_c
\]  
où \(\oplus\) représente la concaténation après redimensionnement du contig maître pour remplacer la "pointe" éventuellement erronée par la séquence propre du candidat. Le contig candidat \(C_c\) est alors marqué comme absorbé.  
Cette recherche est effectuée à la fois sur \(C_m\) dans le sens direct et sur son complément inverse \(C_m^\text{RC}\) afin de gérer les contigs qui auraient été assemblés dans la direction opposée.

Ces contigs sont ensuite exploités par la fonction \texttt{exportToGFA()} (que nous ne détaillons pas ici, car nous devons faire des choix)  qui produit un fichier GFA conforme où chaque contig apparaît comme un segment annoté par sa couverture. Cette représentation permet de visualiser efficacement l’assemblage, notamment avec des outils tels que \texttt{Bandage}.

\subsection{Outils auxiliaires \textit{(Exercice 7)}}
    
\texttt{Quast} (QUality ASsessment Tool), dans un premier temps, nous permet d’évaluer la qualité d’un assemblage à l’aide d’un ensemble de statistiques standardisées (longueur totale, taux d’erreurs, fragments mal alignés, etc.). Bien que l’outil puisse fonctionner en mode de novo, sans référence, nous avons choisi d’utiliser ici la séquence de référence fournie pour la mitochondrie du varan de Komodo. Ce choix garantit une évaluation plus robuste en ancrant nos comparaisons sur un génome connu, ce qui maximise la pertinence\\

Dans un second temps, nous avons exécuté \texttt{minia}, un autre algorithme d’assemblage à base de graphe de Bruijn, afin d’obtenir un assemblage alternatif. Cet assemblage sert de point de comparaison directe avec celui obtenu précédemment.\\

Enfin, nous avons utilisé \texttt{D-genies}, qui génère un dot-plot visualisant le degré d’identité entre deux séquences. Cet outil complète l’analyse en offrant une représentation intuitive de la synténie, des éventuelles réarrangements, et des discordances locales entre notre assemblage principal et celui obtenu avec \texttt{minia}. L’approche par dot-plot complète notre stratégie puisqu'elle combine des métriques numériques (via Quast) et une inspection visuelle qualitative (via D-genies), ce qui nous à permis de valider nos choix d'implémentations après avoir identifier de grande divergences dans la nature et le nombre de contigs obtenu (dans nos premières exécutions du programme). 

\newpage

\section{Résultats \& discussion \textit{(Exercice 6 \& 7)}}
\subsection{Impact des étapes de simplification}
Sans l'étape de fusion présentée plus haut, le graphe produit de nombreux contigs courts, localement cohérents mais fragmentés, et souvent recouvrants. La fusion rassemble ces segments et fait apparaître un contig unique, plus représentatif de la séquence attendue.\\ 

Dans notre cas d'étude, l'élagage des \textit{tips} et la résolution des bulles n'ont eu (finalement) qu'un impact minimal~: ils réduisent le graphe de 147 à 146 contigs, qui est dans notre cas, un contig compris dans un plus grand. Pour cette instance du problème, ces étapes de simplification ne sont donc pas déterminantes. Pour autant, on pourrait aller plus loin en examinant un plus grand nombre de cas limites et en soumettant des fichiers de lecture de plus grande taille afin de vérifier si cette observation se généralise. C’est un point que nous n’avons malheureusement pas eu le temps d’explorer de manière satisfaisante. \\

Inversement, la génération des chemins linéaires suivie de la fusion par  \og Deep Seeding \fg{} suffit à retrouver un contig continu. Cette étape corrige  naturellement les fragments qui se chevauchent, sans dépendre des heuristiques de nettoyage du graphe. Pour garantir une évaluation honnête, nous avons désactivé l’élagage des \textit{tips} et la résolution des bulles durant les benchmarks. Ces opérations prennent trop de temps pour un résultat similaire dans notre cas biologique. Leur retrait permet donc de mesurer uniquement l’effet réel du couple \og génération + fusion \fg{}. Ce que vous pouvez constater dans la figure 3 ci-dessous :
\begin{figure}[!h]
    \centering
    % Première image (Minia)
    \begin{subfigure}[b]{0.49\textwidth}
        \centering
        \includegraphics[width=\linewidth]{IMAGES/ramilass_contigs_unfused_varankomodo_reference.png}
        \caption{\underline{Ramilass Contigs sans fusion}}
        \label{fig:minia}
    \end{subfigure}
    \hfill % Espace flexible entre les deux images
    % Deuxième image (Ramilass)
    \begin{subfigure}[b]{0.49\textwidth}
        \centering
        \includegraphics[width=\linewidth]{IMAGES/ramilass_contigs_fused_varankomodo_reference.png}
        \caption{\underline{Ramilass Contigs avec fusion}}
        \label{fig:ramilass}
    \end{subfigure}
    
    \caption{Comparaison des alignements avec (a) et sans (b) fusion par rapport à la référence Varankomodo}
    \label{fig:comparaison_alignements}
\end{figure}

\newpage
\subsection{(Evaluation quantitative et qualité de l’assemblage \textit{(Exercice 7)}}

\begin{wraptable}{r}{0.52\textwidth}
\vspace{-1em}
\centering
\caption{\textbf{Statistiques d'assemblage : Minia vs Ramilass Fuse}}
\label{tab:assemblage}
\rowcolors{2}{white!95!bleumarine!5}{white}
\begin{tabular}{lrr}
\toprule
\textbf{Statistiques} & \textbf{Minia} & \textbf{Ramilass} \\
\midrule
\multicolumn{3}{l}{\textit{Référence : longueur = 10 624 bp, GC = 44.14\%}} \\
\midrule
Nombre de contigs ($\ge 0$ bp)       & 4     & 3     \\
Nombre de contigs ($\ge 1000$ bp)    & 1     & 1     \\
Longueur totale (bp)                  & 10 187 & 10 301 \\
Plus grand contig (bp)                & 9 936  & 10 094 \\
GC (\%)                               & 44.05 & 44.03 \\
Mismatches / 100 kbp                  & 0     & 89.16 \\
Indels / 100 kbp                      & 0     & 9.91 \\
\bottomrule
\end{tabular}
\vspace{-1em}
\end{wraptable}

La comparaison entre \texttt{Minia} et \texttt{Ramilass} révèle que ce dernier génère un contig principal plus long et plus proche de la référence (cf. table 1). Bien que légèrement plus déréglé par les erreurs ($\approx 89$/100 kbp), le profil GC reste très proche de la référence (44.03\%), ce qui signe une reconstruction biologique cohérente. Le taux d’erreur correspond à un score Phred d’environ 30, un seuil généralement considéré comme satisfaisant pour l’assemblage.
\vspace{5mm}
\begin{wraptable}{l}{0.52\textwidth}
\vspace{-1em}
\centering
\caption{\textbf{Performances : temps d'exécution et mémoire} (table 3)}
\label{tab:benchmark_results}
\rowcolors{2}{white!95!bleumarine!5}{white}
\begin{tabular}{lcc}
\toprule
\textbf{Outil} & \textbf{Temps (s)} & \textbf{Mémoire (MB)} \\
\midrule
Minia    & $0.59 \pm 0.02$ & $249.7 \pm 1.2$ \\
Ramilass & $0.32 \pm 0.00$ & $34.2 \pm 0.2$ \\
SPAdes   & $4.93\pm 0.10$ & $69.9 \pm 0.2$ \\
\bottomrule
\end{tabular}
\vspace{-1em}
\end{wraptable}

Les mesures de performance confirment l’intérêt du modèle proposé (cf. table 2). L’assembleur s’exécute plus rapidement (0,32 s) et consomme beaucoup moins de mémoire (34 Mo) que \texttt{Minia} (249 Mo) ou \texttt{SPAdes} (69 Mo).  
Cette efficacité découle d’une représentation compacte par \textit{bitvectors}, d’un traitement minimal des simplifications et d’une fusion basée sur les \textit{seeds} plutôt que sur une reconstruction complète du graphe.  
La faible variabilité des mesures montre un comportement stable et reproductible, assurant un assemblage proche de la référence tout en minimisant les coûts computationnels.\\

Nous rajouterons avoir utilisé \texttt{Valgrind} pour vérifier l’absence de fuites mémoire. Les résultats confirment que toutes les allocations sont correctement libérées, cet outil découvert en DevOps nous à paru approprié dans le contexte de ce projet, exemple pour une instance de test : {total heap usage: 215,529 allocs, 215,529 frees}.

\subsection{Limites identifiées \& perspectives d'améliorations \textit{(Exercice 6)}}

Nous vous listons/proposons quelques unes des critiques, assurément, non exhaustives de notre travail, ci-dessous.

\textbf{\underline{Prise en charge d'un alphabet limité :}} tout d'abord, rappelons que la gestion des fichiers se limite aux nucléotides A, T, C et G. Les lettres dégénérées du standard \texttt{IUPAC} telles que N, R ou Y sont ignorées, ce qui peut poser problème dans le cadre de fichiers de qualité médiocre, hétérogènes.\\

\textbf{\underline{L'apport de données pairées :}} Une autre limite est les modalités de génération de nos chemins linéaires dans le graphe qui peut conduire à la formation de contigs chimériques. L’intégration de données pairées pourrait limiter ce phénomène et améliorer la résolution de structures complexes. En effet, nous avons remarquer que des amas complexes de \textit{tips} et bulles apparaissent (\autoref{fig:resolveBubbles_complex}), difficiles à résoudre sans heuristique fine.\\ 

\textbf{\underline{La perte d'information haplotypique :}} Nous savons également que pour des génomes polyploïdes, en particulier homozygotes, la fusion des contigs, dans la majorité des assembleurs \textit{de novo}, tend à condenser plusieurs haplotypes en un seul, masquant de la diversité génétique, notre outil ne fera pas exception à cette limite. \\ 

\textbf{\underline{Un compromis qualité/vitesse:}} La stratégie que nous avons adoptée de fusion \og par pas de k-mers \fg  plutôt qu’un nucléotide par nucléotide sacrifie légèrement la qualité pour gagner en vitesse, accélérant l’assemblage certes, mais pouvant introduire de petites erreurs de jonction. \\

\textbf{\underline{La scalabilité}} reste, selon nous, le principal facteur limitant. Bien que l'outil soit performant sur de petits virus ou bactéries, il est probable qu'il échoue sur un génome humain (\(3 \times 10^9\) bases). Trois goulots majeurs ont été identifiés : la taille et le coût mémoire de la structure \texttt{Noeud} (\autoref{lst:noeud_struct}), la fragmentation due aux millions de micro-allocations via \texttt{new}, et la double indirection introduite par \texttt{std::vector<Noeud*>}. Cette dernière entraîne un accès indirect aux données : il faut d'abord lire le vecteur pour récupérer l'adresse du nœud enfant (\texttt{Noeud*}), puis suivre ce pointeur pour accéder aux champs de l'objet. Ce schéma, classique en C++, provoque une surcharge mémoire et dégrade l'efficacité pour de grands volumes de données.

\subsection{Un brin de complexité}

Notre programme se décompose en plusieurs étapes successives dont la complexité globale dépend principalement de \underline{la taille de nos lectures}, du \underline{nombre de k-mers} et du \underline{nombre de contigs}. Nous pensons que la lecture du fichier \texttt{FASTA} et la construction du \textcolor{blue!80!black}{\texttt{BitVector}} se font en temps linéaire par rapport au nombre total de nucléotides, $O(\textcolor{blue!80!black}{N})$ car chaque nucléotide est traité exactement une fois. \\

La création du graphe de kmers via \textcolor{red!80!black}{\texttt{Graphdbj}} suit également une logique linéaire, $O(\textcolor{red!80!black}{N})$, puisque proportionnelle au nombre de k-mers générés. Les étapes de simplification, \textcolor{green!60!black}{\texttt{resolveBubbles}} et \textcolor{green!60!black}{\texttt{clipTips}}, ont une complexité $O(\textcolor{green!60!black}{P} \cdot (\textcolor{green!60!black}{V}+\textcolor{green!60!black}{E}))$, avec \textcolor{green!60!black}{P} le nombre de passes et \textcolor{green!60!black}{V}, \textcolor{green!60!black}{E} le nombre de noeuds et d’arêtes, car chaque passe examine potentiellement dans le pire des cas tous les éléments du graphe brut. Ensuite, la comparaison des kmers entre lectures, effectuée par \textcolor{violet!80!black}{\texttt{CompareKMers}}, est en théorie quadratique en nombre de lectures et linéaire en taille des k-mers, $O(\textcolor{violet!80!black}{R}^2 \cdot \textcolor{violet!80!black}{k})$, puisqu'on doit comparer chaque k-mer à tous ses compères. Mais en pratique on peut dire que seule la portion initiale des k-mers est réellement comparée pour la fusion, justifiant notre choix de ne pas considérer toutes les positions possibles dans le calcul théorique. La fusion des contigs, toujours via \textcolor{violet!80!black}{\texttt{CompareKMers}}, est quadratique en nombre de contigs, $O(\textcolor{violet!80!black}{C}^2 \cdot \textcolor{violet!80!black}{k})$. Enfin, l’export des contigs en FASTA ou GFA via \textcolor{black!70}{\texttt{Convert}} est proportionnel à la somme des longueurs des contigs, $O(\textcolor{black!70}{S})$. En combinant les étapes dominantes que nous venons d'expliciter, la complexité théorique dans le pire des cas est donc :

\[
\textcolor{blue!80!black}{O(\textcolor{blue!80!black}{N})} + 
\textcolor{red!80!black}{O(\textcolor{red!80!black}{N})} + 
\textcolor{green!60!black}{O(\textcolor{green!60!black}{P}(\textcolor{green!60!black}{V}+\textcolor{green!60!black}{E}))} + 
\textcolor{violet!80!black}{O(\textcolor{violet!80!black}{R}^2 \textcolor{violet!80!black}{k} + \textcolor{violet!80!black}{C}^2 \textcolor{violet!80!black}{k})} + 
\textcolor{black!70}{O(\textcolor{black!70}{S})} 
= O(\textcolor{blue!80!black}{N} + \textcolor{green!60!black}{P}(\textcolor{green!60!black}{V}+\textcolor{green!60!black}{E}) + \textcolor{violet!80!black}{R}^2 \textcolor{violet!80!black}{k} + \textcolor{violet!80!black}{C}^2 \textcolor{violet!80!black}{k} + \textcolor{black!70}{S})
\]

En synthèse, nous proposons comme ordre de grandeur de Landau pour la complexité temporelle de notre algorithme : 

\[
\boxed{\textcolor{violet!80!black}{O_\text{max}} = O(\textcolor{violet!80!black}{\max(\textcolor{violet!80!black}{R}^2 \textcolor{violet!80!black}{k}, \textcolor{violet!80!black}{C}^2 \textcolor{violet!80!black}{k})})}
\]

Ajoutons que $(\textcolor{violet!80!black}{R}^2 \cdot \textcolor{violet!80!black}{k})$ domine si le nombre de lectures est grand et que $k$ est conséquent. Mais que $( \textcolor{violet!80!black}{C}^2 \cdot \textcolor{violet!80!black}{k})$ domine si le nombre de contigs est grand et que k est conséquent, donc l'ordre de grandeur synthétique global n'est pas tout à fait $\mathcal{O}(N)^2$.


\subsection{Conclusion générale}

Lors de ce TP nous avons réussit a réaliser un petit assembleur pour permettant d'assembler des \textit{short reads} d'individus haploïde ou d'ADN mitochondriaux/chloroplastique. Ce dernier utilise des méthodes bien documenté dans la littérature ainsi que des optimisations en mémoire originales ce qui lui permet d'être a vitesse équivalente, moins gourmand en mémoire et assemblant des contigs plus long que \texttt{minia} pour notre problème biologique. Nous avons entièrement décrit le fonctionnement de notre logiciel avec la description de ses complexités en temps et en mémoire ainsi que la détermination de ses limitations. 

Vous retrouverez l’entièreté du code source sur ce répertoire \hyperlink{https://github.com/MickaelCQ/Assembleur}{Github}. Le projet a été entièrement conçu en privilégiant la reproductibilité, d'où l'intégration de Pixi et la possibilité d'exécuter notre outil dans un conteneur Singularity/Apptainer.
\appendix
\renewcommand{\thesection}{\Alph{section}}

\addtocontents{toc}{
  \protect\renewcommand{\protect\cftsecpagefont}{}%
  \protect\renewcommand{\protect\cftsubsecpagefont}{}%
  \protect\renewcommand{\protect\cftsecafterpnum}{}%
  \protect\renewcommand{\protect\cftsubsecafterpnum}{}%
}
\section{Annexes}
\subsection*{Etat de l'art : algorithme Glouton \textit{(Exercice 1)}}
\label{algo1Glouton}
\begin{center}
\begin{algorithm}[H]
\textcolor{bleumarine}{\textbf{\caption{\textcolor{bleumarine}{Assemblage\_Glouton()}}}}
\KwEntree{Famille de mots \( F \) de taille \( n \), où \( L \) est la longueur moyenne des mots}
\KwSortie{Chaîne de caractères \( S \) assemblée}
\KwComplexite{Temps : \(\mathcal{O}(n^3 + n^2 L)\), Espace : \(\mathcal{O}(n^2)\)}
\KwDebut \\
    \Indp
     \( S \gets \text{""} \) \textcolor{black!50} ; \( M \gets \) tableau d'entiers de taille \( n \times n \) \textcolor{black!50}{~($\mathcal{O}(n^2)$)}\;
    \KwPour \( i \) allant de \( 0 \) à \( n-1 \) \KwFaire \textcolor{black!50}{~($\mathcal{O}(n^2 L)$)}\\
        \Indp
        \KwPour \( j \) allant de \( 0 \) à \( n-1 \) \KwFaire \textcolor{black!50}{~($\mathcal{O}(n L)$)}\\
            \Indp
            \( M[i][j] \gets \text{longueur du chevauchement maximal entre } F[i] \text{ et } F[j] \) \textcolor{black!50}{~($\mathcal{O}(L)$)}\;
            \Indm
        \KwFinPour \\
        \Indm
    \KwFinPour \\
    \( m \gets 0 \) \textcolor{black!50}\;
    \KwTantQue \( m < n-1 \) \KwFaire \textcolor{black!50}{~($\mathcal{O}(n)$)}\\
        \Indp
        \( \textit{imax} \gets 0 \); \( \textit{jmax} \gets 0 \); \( \textit{max} \gets M[0][0] \) \textcolor{black!50}\;
        \KwPour \( i \) allant de \( 0 \) à \( n-1 \) \KwFaire \textcolor{black!50}{~($\mathcal{O}(n^2)$)}\\
            \Indp
            \KwPour \( j \) allant de \( 0 \) à \( n-1 \) \KwFaire \textcolor{black!50}{~($\mathcal{O}(n)$)}\\
                \Indp
                \KwSi \( \textit{max} < M[i][j] \) \KwAlors \textcolor{black!50}\\
                    \Indp
                    \( \textit{imax} \gets i \); \( \textit{jmax} \gets j \); \( \textit{max} \gets M[i][j] \) \textcolor{black!50}\;
                    \Indm
                \KwFinSi \\
                \Indm
            \KwFinPour \\
            \Indm
        \KwFinPour \\
        \( S \gets S + F[\textit{imax}][0:\text{longueur}(F[\textit{imax}]) - \textit{max}] \) \textcolor{black!50}{~($\mathcal{O}(L)$)}\;
        \( S \gets S + F[\textit{jmax}] \) \textcolor{black!50}{~($\mathcal{O}(L)$)}\;
        \KwPour \( i \) allant de \( 0 \) à \( n-1 \) \KwFaire \textcolor{black!50}{~($\mathcal{O}(n)$)}\\
            \Indp
            \( M[\textit{imax}][i] \gets -1 \); \( M[i][\textit{jmax}] \gets -1 \) \textcolor{black!50}\;
            \Indm
        \KwFinPour \\
        \( m \gets m + 1 \) \textcolor{black!50}\;
        \Indm
    \KwFinTantQue \\
    \KwRetourner \( S \) \textcolor{black!50}\;
\Indm
\KwFin
\end{algorithm}
\end{center}

\subsection*{Etat de l'art : construction du graphe de chevauchement \textit{(Exercice 1)}}

\begin{center}
\begin{algorithm}[H]
\textcolor{bleumarine}{\textbf{\caption{Build\_Graphe\_Chevauchement()}}}
\KwEntree{Famille de mots $\mathcal{F}$ , où chaque \( F_i \) est un mot}
\KwSortie{Graphe de chevauchement \( G = (V, E) \), où \( V \) est l'ensemble des sommets (chaque sommet représente un mot de \( F \)) et \( E \) est l'ensemble des arêtes pondérées par la longueur du chevauchement}
\KwComplexite{Temps : \(\mathcal{O}(n^2 L)\), Espace : \(\mathcal{O}(n^2)\) où \( L \) est la longueur moyenne des mots}
\KwDebut \\
    \Indp
    \( V \gets F \) \;
    \( E \gets "" \) \;
    \KwPour \( i \) allant de \( 0 \) à \( n-1 \) \KwFaire \textcolor{black!50}{~($\mathcal{O}(n^2 L)$)}\\
        \Indp
        \KwPour \( j \) allant de \( 0 \) à \( n-1 \) \KwFaire \textcolor{black!50}{~($\mathcal{O}(n L)$)}\\
            \Indp
            \KwSi \( i \neq j \) \KwAlors \\
                \Indp
                \( k \gets \text{longueur du chevauchement maximal entre } F[i] \text{ et } F[j] \) \textcolor{black!50}{~($\mathcal{O}(L)$)}\;
                \KwSi \( k > 0 \) \KwAlors \\
                    \Indp
                    \( E \gets E \cup \{(F[i], F[j], k)\} \) \;
                    \Indm
                \KwFinSi \\
                \Indm
            \KwFinSi \\
            \Indm
        \KwFinPour \\
        \Indm
    \KwFinPour \\
    \KwRetourner \( G = (V, E) \) \;
\Indm
\KwFin
\end{algorithm}
\end{center}


\begin{center}
\begin{figure}
\begin{algorithm}[H]
\textcolor{bleumarine}{\textbf{\caption{Assemblage\_Graphe\_Chevauchement()}}}
\KwEntree{Famille de mots $\mathcal{F}$; graphe de chevauchement \( G = (V, E) \) construit à partir de $\mathcal{F}$}
\KwSortie{Chaîne de caractères \( S \) assemblée}
\KwComplexite{Temps : \(\mathcal{O}(n^2 L + n^3)\), Espace : \(\mathcal{O}(n^2)\)}
\KwDebut \\
    \Indp
    " Initialisation"\\
    \( S \gets \text{""} \) \;
    \( \text{Visite} \gets \) tableau de booléens de taille \( n \), initialisé à \(\text{FAUX}\) \;
    \( \text{chemin} \gets \) liste vide \;
    \( \text{degré\_entrant} \gets \) tableau d'entiers de taille \( n \), initialisé à \( 0 \) \;

    % Calcul des degrés entrants pour chaque sommet
    \KwPour chaque arête \( (u, v, k) \in E \) \KwFaire \\
        \Indp
        \( \text{degré\_entrant}[v] \gets \text{degré\_entrant}[v] + 1 \) \;
        \Indm
    \KwFinPour

    % Sélection du sommet de départ (celui avec le plus petit degré entrant)
    \( u \gets \) sommet avec le plus petit \(\text{degré\_entrant}\) \;

    % Construction du chemin en suivant les arêtes de poids maximal
    \KwTantQue \( u \neq \text{AUCUN} \) \KwFaire \\
        \Indp
        \( \text{chemin}.ajouter(u) \) \;
        \( \text{Visite}[u] \gets \text{VRAI} \) \;
        % Trouver le sommet non Visite avec le chevauchement maximal
        \( v \gets \) sommet non Visite tel que \( (u, v, k) \in E \) et \( k \) est maximal \;
        \( u \gets v \) \;
        \Indm
    \KwFinTantQue

    % Construction de la superchaîne \( S \) à partir du chemin
    \KwSi \( \text{chemin.taille} > 0 \) \KwAlors \\
        \Indp
        \( S \gets F[\text{chemin}[0]] \) \;
        \KwPour \( i \) allant de \( 1 \) à \( \text{chemin.taille} - 1 \) \KwFaire \\
            \Indp
            \( k \gets \) longueur du chevauchement entre \( F[\text{chemin}[i-1]] \) et \( F[\text{chemin}[i]] \) \;
            \( S \gets S + F[\text{chemin}[i]][k:] \) \;
            \Indm
        \KwFinPour \\
        \Indm
    \KwFinSi

    \KwRetourner \( S \) \;
\Indm
\KwFin
\end{algorithm}
\caption{\underline{Etat de l'art :Algorithme de construction du graphe de chevauchement, comme étudié en cours}}
\end{figure}
\end{center}

\newpage
\subsection*{Spécifications du Système pour les benchmarks}
\begin{table}[h!]
    \centering
    \begin{tabular}{l l}
        \toprule
        \textbf{Catégorie} & \textbf{Détail} \\
        \midrule
        \multicolumn{2}{l}{\textbf{Système}} \\
        Système d'Exploitation & \textbf{Ubuntu 24.04.3 LTS} \\
        Noyau (Kernel) & \textbf{6.14.0-36-generic} \\
        \midrule
        \multicolumn{2}{l}{\textbf{Processeur (CPU)}} \\
        Modèle & \textbf{AMD Ryzen 7 7730U with Radeon Graphics} \\
        Architecture & $\text{x86\_64}$ \\
        Cœurs physiques | Threads & $\text{8 cœurs} \mid \text{16 threads}$ \\
        Fréquence Max. & $4547 \text{ MHz}$ \\
        \midrule
        \multicolumn{2}{l}{\textbf{Mémoire Vive (RAM)}} \\
        Taille Totale & $\text{15 GiB}$ \\
        \midrule
        \multicolumn{2}{l}{\textbf{Graphique (GPU)}} \\
        Contrôleur | Modèle & \textbf{Advanced Micro Devices, Inc. [AMD/ATI]} \\
        Produit (Puce Intégrée) & \textbf{Barcelo (AMD Radeon Graphics)} \\
        \bottomrule
    \end{tabular}
    \label{tab:specs}
    \caption{Spécifications du Système - Ubuntu}
\end{table}


\newpage 
\subsection*{Exemple d'une structure complexe : cas limite de notre algorithme \textit{(Exercice 6 )}}
\begin{figure}[h!]
    \centering
    \includegraphics[width=1\textwidth]{IMAGES/HAAAAAAAAAAAAAA.png}
    \caption{\underline{Visualisation \texttt{Bandage} d'une structure non résolue par l'algorithme.}}
    \label{fig:resolveBubbles_complex}
\end{figure}


\usetikzlibrary{shapes.geometric, arrows, positioning, calc, fit, backgrounds}

\tikzset{
    startstop/.style={rectangle, rounded corners, minimum width=3cm, minimum height=1cm, text centered, draw=black, fill=blue!10},
    process/.style={rectangle, minimum width=3cm, minimum height=1cm, text centered, draw=black, fill=green!10},
    decision/.style={diamond, aspect=2, minimum width=3cm, minimum height=1cm, text centered, draw=black, fill=orange!10},
    arrow/.style={thick,->,>=stealth'},
    block/.style={rectangle, minimum width=3cm, minimum height=1cm, text width=5cm, text centered, draw=black, fill=gray!10},
    substep/.style={rectangle, minimum width=2.5cm, minimum height=0.8cm, text width=5cm, align=center, draw=black!50, fill=white},
    % Nouveaux styles pour les cadres de fond
    phasebox/.style={draw, dashed, rounded corners, inner sep=0.5cm, line width=1pt}
}

\begin{tikzpicture}[node distance=1.2cm, auto, scale=0.7, transform shape]

    % --- NOEUDS ---
    \node (start) [startstop] {Début \texttt{mergeContigs}};
    \node (global_loop_init) [process, below=of start] {Init \texttt{global\_change = true}};
    \node (global_loop_decision) [decision, below=of global_loop_init] {\texttt{global\_change}?};
    \node (end) [startstop, right=of global_loop_decision, xshift=2.5cm] {Fin : Retourner contigs};
    \node (global_loop_reset) [process, below=of global_loop_decision, yshift=-0.5cm] {Reset \texttt{false}};

    % --- PHASE 1 ---
    \node (phase1_label) [block, below=of global_loop_reset] {PHASE 1 : Containment};
    \node (phase1_step1) [substep, below=of phase1_label] {1. Indexation Débuts\\ (start\_map)};
    \node (phase1_step2) [substep, below=of phase1_step1] {2. Scan Inclusion\\ (absorbed)};
    
    % --- PHASE 2 ---
    \node (phase2_label) [block, below=of phase1_step2, yshift=-0.8cm] {PHASE 2 : Extension};
    \node (phase2_step1) [substep, below=of phase2_label] {1. Indexation Graines\\ (seed\_map)};
    \node (phase2_loop_start) [process, below=of phase2_step1] {Pour chaque Maître non absorbé};
    
    \node (try_std_ext) [process, below=of phase2_loop_start] {Tentative 1 : Extension Std};
    \node (std_ext_success) [decision, below=of try_std_ext] {Succès ?};
    \node (merge_success_std) [process, below=of std_ext_success, yshift=-0.5cm] {Change=true, Maître étendu};

    \node (try_rc_ext) [process, right=of std_ext_success, xshift=3cm] {Tentative 2 : Extension RC};
    \node (rc_ext_success) [decision, below=of try_rc_ext] {Succès ?};
    \node (merge_success_rc) [process, below=of rc_ext_success, yshift=-0.5cm] {Change=true, Remplacement RC};

    \node (next_master) [process, below=of merge_success_std, yshift=-1.5cm] {Maître suivant};
    \node (end_loop_phase2) [coordinate, below=of next_master, yshift=-0.5cm] {};

    % --- FLÈCHES ---
    \draw [arrow] (start) -- (global_loop_init);
    \draw [arrow] (global_loop_init) -- (global_loop_decision);
    \draw [arrow] (global_loop_decision) -- node[above] {Non} (end);
    \draw [arrow] (global_loop_decision) -- node[right] {Oui} (global_loop_reset);
    \draw [arrow] (global_loop_reset) -- (phase1_label);
    \draw [arrow] (phase1_label) -- (phase1_step1);
    \draw [arrow] (phase1_step1) -- (phase1_step2);
    \draw [arrow] (phase1_step2) -- (phase2_label);
    \draw [arrow] (phase2_label) -- (phase2_step1);
    \draw [arrow] (phase2_step1) -- (phase2_loop_start);
    \draw [arrow] (phase2_loop_start) -- (try_std_ext);
    \draw [arrow] (try_std_ext) -- (std_ext_success);
    
    \draw [arrow] (std_ext_success) -- node[right] {Oui} (merge_success_std);
    \draw [arrow] (std_ext_success) -- node[above] {Non} (try_rc_ext);
    
    \draw [arrow] (try_rc_ext) -- (rc_ext_success);
    \draw [arrow] (rc_ext_success) -- node[right] {Oui} (merge_success_rc);
    
    % Flèche NON qui contourne large
    \draw [arrow] (rc_ext_success.east) -- node[above] {Non} ++(2,0) |- (next_master.east);

    \draw [arrow] (merge_success_std) -- (next_master);
    \draw [arrow] (merge_success_rc.south) |- (next_master);

    \draw [arrow] (next_master) -- (end_loop_phase2);
    \draw [arrow] (end_loop_phase2) -- ++(-7,0) node[midway, above] {Cycle suivant} |- ([yshift=0.5cm]global_loop_decision.north) -- (global_loop_decision.north);

    % --- CADRES DE FOND (BACKGROUNDS) ---
    % On utilise un layer pour dessiner DERRIÈRE les noeuds existants
    \begin{pgfonlayer}{background}
        
        % Cadre Phase 1 : Englobe le label Phase 1 et l'étape 2
        \node [fit=(phase1_label) (phase1_step2), 
               phasebox, fill=blue!5, draw=blue!30, 
               label={[anchor=north west, inner sep=5pt, color=blue!80]north west:\textbf{PHASE 1}}] {};

        % Cadre Phase 2 : Englobe le label Phase 2, l'extension RC (tout à droite) et le maitre suivant (tout en bas)
        \node [fit=(phase2_label) (try_rc_ext) (merge_success_rc) (next_master) (phase2_loop_start), 
               phasebox, fill=red!5, draw=red!30, 
               label={[anchor=north west, inner sep=5pt, color=red!80]north west:\textbf{PHASE 2}}] {};
               
    \end{pgfonlayer}

\end{tikzpicture}


\begin{table}[h!]
    \centering
        \caption{Répartition du temps d'exécution optimisé de l'Assembleur GraphDBJ.}
    \label{tab:execution_time_new}
    \begin{tabular}{lcc}
        \toprule
        \textbf{Étape} & \textbf{Temps (ms)} & \textbf{Pourcentage du Total} \\
        \midrule
        Création du Graphe & 175 & $53.8\%$ \\
        Fusion des Contigs & 100.5 & $30.9\%$ \\
        Écriture Fichiers & 36 & $11.1\%$ \\
        Génération Contigs & 9.5 & $2.9\%$ \\
        Lecture FASTA & 4.5 & $1.4\%$ \\
        Simplification & 0 & $0.0\%$ \\
        \bottomrule
        \textbf{TOTAL} & $\mathbf{325.5}$ & $\mathbf{100.0\%}$ \\
        \bottomrule
    \end{tabular}
\end{table}

\section{Références}
\vspace{3mm}
\subsection*{Bibliographie}
\printbibliography[heading=none]

\subsection*{Webographie :}
\textbf{QUAST : }
\url{https://github.com/ablab/quast}\\
\textbf{D-GENIES : }\url{https://dgenies.toulouse.inra.fr/}\\
\textbf{Minia : }\url{https://gatb.inria.fr/software/minia/}\\
\textbf{SPades : }\url{https://github.com/ablab/spades}\\


\renewcommand{\bibname}{Bibliographie compilée}  

\end{document}